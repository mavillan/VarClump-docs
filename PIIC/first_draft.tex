\documentclass[spanish, fleqn]{article}
\usepackage[spanish]{babel}
\usepackage[utf8]{inputenc}
\usepackage{amsmath}
\usepackage{amsfonts,txfonts}
\usepackage{mathrsfs}
\usepackage[colorlinks, urlcolor=blue]{hyperref}
\usepackage{fourier}
\usepackage[top = 2.5cm, bottom = 2cm, left = 2cm, right = 2cm]{geometry}
\usepackage{graphicx}




\title{Identificación de Clumps Astronómicos: Un Enfoque Desde el Cálculo Variacional}
\author{Martín Villanueva}
\date{4 de abril de 2016}

\begin{document}
\maketitle


\section{Introducción}
Un cubo espectroscópico de datos puede ser modelado por una función $f(x,y,z)$, donde
las variables $(x,y)$ codifican las coordenadas espaciales (\textit{right ascension} y
\textit{declination}) respectivamente, y la coordenada $z$ corresponde a la velocidad
radial. Esta función $f$ asocia una de estas coordenadas, un valor correspondiente a la
intensidad de la emisión captada en tal punto. Sin embargo la función $f$ se conoce en
un conjunto discreto de puntos (valores en el cubo de datos).

El problema entonces reside en encontrar una función $g(x,y,z)$ continua, que aproxime
la función $f$, de tal manera que estudiando la forma y propiedades de $g$, sea posible
identificar clumps o regiones de interés en dichas observaciones astronómicas.  

\section{El modelo Variacional}

Sea entonces $f(x,y,z)$ la función de data, y $g(x,y,z)$ la función que pretende aproximar
a $f$, se propone entonces el siguiente funcional:

\begin{equation} \label{eq1}
\begin{split}
\Phi(g) & = \int_{\Omega \subset R^3} L(x, y , z , g , g_x, g_y, g_z) d\Omega \\
        & = \int_{\Omega \subset R^3} (f-g)^2 +\underbrace{\alpha e^{-(f-g)}}_{\text{Penalty term}}
        + \underbrace{\beta \ \Psi(g_x, g_y, g_z)}_{\text{Smooth term}} \ d\Omega
\end{split}
\end{equation}

\noindent a ser minimizado. La inclusión del \textit{Penalty term} y \textit{Smooth term} en el funcional
tiene los siguientes motivos:

\begin{enumerate}
    \item \textit{Penalty term}: Una de los requisitos sobre la función $g$ es que esta nunca \textit{sobrepase}
    a $f$. Luego agregando este término se genera una alta penalización sobre el funcional cuando $g(x,y,z) > f(x,y,z)$.
    Esto a su vez puede ser controlado por el parámetro $\alpha$.
    \item \textit{Smooth term}: Siendo $\Psi$ una función continua, monótona y creciente, el objetivo de este término
    es agregar \textit{smoothness} a la solución $g$, permitiendo de esta manera filtrar el ruido o distorsiones presentes
    en la data. El parámetro $\beta$ permite controlar el peso de este término, y por lo tanto debe estar directamente
    relacionado con el SNR (\textit{Signal to Noise Ratio}).
\end{enumerate}

Ocupando entonces los resultados del cálculo variacional, sabemos que el funcional $\Phi(g)$ es minimizado cuando
se satisface la siguiente ecuación de Euler-Lagrange:

\begin{equation}
    \frac{\partial L}{\partial g} - \frac{d}{dx}\frac{\partial L}{\partial g_x}
    - \frac{d}{dy}\frac{\partial L}{\partial g_y} - \frac{d}{dz}\frac{\partial L}{\partial g_z} = 0
    \ \ \ \forall (x,y,z) \in \Omega
\label{eq1}
\end{equation}

\section{Resolución del Problema}

Como primer acercamiento a la solución de (\ref{eq1}), se propone el método de colocación. Sea $\mathcal{N}(\mathbf{x},\mathbf{x_0},\sigma)$ una función Gaussiana unitaria, centrada en $x_0$ con desviación estándar $\sigma$, entonces la función $g$ propuesta es
como sigue:

\begin{equation}
     g(\mathbf{x}) = \sum_{k=1}^N \gamma_k \mathcal{N}(\mathbf{x}, \mathbf{x}_k, \sigma_k)
 \label{eq2}
 \end{equation}

 \noindent que corresponde a una combinación lineal de tales Gaussianas. Cada uno de los componentes de la combinación lineal tiene dos parámetros a determinar ($\gamma_k, \sigma_k$).

 \

 \textbf{Observación}: Eventualmente pueden ser usadas Gaussianas elípticas, asumiendo que el número de parámetros a
 determinar será mayor: Rotaciones de los ejes, y desviaciones estándar a lo largo de cada semieje.

\

El problema se resume entonces en encontrar la configuración de $(\gamma_k, \sigma_k)$, de modo que
$\Phi(g)$ sea minimizado. Reemplazando (\ref{eq2}) en (\ref{eq1}) entrega (en general) una ecuación no lineal en
$\{(\gamma_k, \sigma_k)\}_{k=1}^N$. Como estas son $2N$ incógnitas, será necesario evaluar (\ref{eq1}) en la misma
cantidad de puntos en su dominio $\Omega$, generando un sistema no-lineal de $N\times N$.

\section{Desafíos}

Se listan a continuación un serie de problemáticas en cuanto al modelo y su implementación

\begin{enumerate}
    \item Penalizar correctamente los casos donde $g>f$.
    \item Establecer una relación directa entre el parámetro $\beta$ y el $SNR$ de la data.
    \item Establecer la cantidad $N$ de Gaussianas a ocupar en la construcción de $g$.
    \item Seleccionar los puntos de colocación $x_k$ (Uniforme, Random, Quasi-Random, Ocupando alguna heurística, etc).
    \item Seleccionar los puntos objetivos donde evaluar (\ref{eq1}).
    \item Usualmente para que el método de colocación sea exitoso, requiere valores de $N$ muy grandes. Esto implíca
    evaluaciones costosas para $g,\ g_x,\ g_y,\ g_z$ las cuales son expresiones que deben computarse en cada evaluación
    de (\ref{eq2}) en algún punto del dominio. Una posible solución a este problema es ocupar FGT (Fast Gauss Transform),
    pero hay que considerar que las funciones $g_x,\ g_y,\ g_z$ no son precisamente sumas de Gaussianas, por lo cual
    modificaciones deben hacerse sobre FGT tradicional.
    \item Debe hallarse una manera eficiente de resolver el sistema no lineal de $N \times N$ resultante. Posibles opciones
    son ocupar el método de Newton (con Jacobiano explícito), o las soluciones estables de un sistema dinámico.  
\end{enumerate}



\end{document}
