\chapter{Introducción}

Uno de los problemas principales  existentes hoy en astronomía, es el desarrollo de mecanismo automatizados para la correcta identificación de clumps, y el análisis estructural de estos en las nubes moleculares.

Cada observación de tales nubes puede contener una cantidad considerable \textit{núcleos}, \textit{cluster} y \textit{clumps}. Hace no mucho tiempo, la tarea de identificación de tales estructuras era realizada manualmente por astrónomos, pero esto ya no es posible debido a los siguientes razones:
\begin{enumerate}
    \item Tales métodos manuales no escalan tan rápido como la generación de nuevos datos, tales como los generados por ALMA fase 2.
    \item El tamaño de las imágenes es enorme y puede contener cientos/miles de estructuras de interés. Por lo que puede requerir una cantidad muy alta de tiempo hacerlo manualmente.
    \item Los cubos espectroscópicos de datos son difíciles de manipular y visualizar al día de hoy. Adicionalmente la resolución en frecuencia de tales cubos es cada vez más fina.
    \item Los resultados de identifación de clumps realizados por astrónomos (manualmente), usualmente no calzan pues entre ellos tienen diferentes concepciones y definiciones acerca de que es un clump y la cómo se relacionan entre ellos (especialmente en los casos de emisiones conjuntas).
\end{enumerate}

La motivación de los métodos computacionales, es poder automatizar estas tareas eliminado el juicio imparcial del astrónomo, y al mismo tiempo poder analizar eficientemente gran cantidad de datos.

En lo que sigue se realiza una descripción formal del problema, definiendo en primer lugar que se entiende por \textit{clump}. Luego se describen las técnicas computacionales ocupadas hoy en día para resolver la problemática de identificación de tales estructuras.  Por último se da una breve descripción del enfoque variacional, y como este ha sido aplicado a otras áreas similares con muy buenos resultados.